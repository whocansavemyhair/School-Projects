\documentclass[a4paper]{article}
\usepackage{geometry}
\usepackage{graphicx}
\usepackage{natbib}
\usepackage{amsmath}
\usepackage{amssymb}
\usepackage{amsthm}
\usepackage{paralist}
\usepackage{epstopdf}
\usepackage{tabularx}
\usepackage{longtable}
\usepackage{multirow}
\usepackage{multicol}
\usepackage[hidelinks]{hyperref}
\usepackage{fancyvrb}
\usepackage{algorithm}
\usepackage{algorithmic}
\usepackage{float}
\usepackage{paralist}
\usepackage[svgname]{xcolor}
\usepackage{enumerate}
\usepackage{array}
\usepackage{times}
\usepackage{url}
\usepackage{fancyhdr}
\usepackage{comment}
\usepackage{environ}
\usepackage{times}
\usepackage{textcomp}
\usepackage{caption}


\urlstyle{rm}

\setlength\parindent{0pt} % Removes all indentation from paragraphs
\theoremstyle{definition}
\newtheorem{definition}{Definition}[]
\newtheorem{conjecture}{Conjecture}[]
\newtheorem{example}{Example}[]
\newtheorem{theorem}{Theorem}[]
\newtheorem{lemma}{Lemma}
\newtheorem{proposition}{Proposition}
\newtheorem{corollary}{Corollary}

\floatname{algorithm}{Procedure}
\renewcommand{\algorithmicrequire}{\textbf{Input:}}
\renewcommand{\algorithmicensure}{\textbf{Output:}}
\newcommand{\abs}[1]{\lvert#1\rvert}
\newcommand{\norm}[1]{\lVert#1\rVert}
\newcommand{\RR}{\mathbb{R}}
\newcommand{\CC}{\mathbb{C}}
\newcommand{\Nat}{\mathbb{N}}
\newcommand{\br}[1]{\{#1\}}
\DeclareMathOperator*{\argmin}{arg\,min}
\DeclareMathOperator*{\argmax}{arg\,max}
\renewcommand{\qedsymbol}{$\blacksquare$}

\definecolor{dkgreen}{rgb}{0,0.6,0}
\definecolor{gray}{rgb}{0.5,0.5,0.5}
\definecolor{mauve}{rgb}{0.58,0,0.82}

\newcommand{\Var}{\mathrm{Var}}
\newcommand{\Cov}{\mathrm{Cov}}

\newcommand{\vc}[1]{\boldsymbol{#1}}
\newcommand{\xv}{\vc{x}}
\newcommand{\Sigmav}{\vc{\Sigma}}
\newcommand{\alphav}{\vc{\alpha}}
\newcommand{\muv}{\vc{\mu}}

\newcommand{\red}[1]{\textcolor{red}{#1}}

\def\x{\mathbf x}
\def\y{\mathbf y}
\def\w{\mathbf w}
\def\v{\mathbf v}
\def\E{\mathbb E}
\def\V{\mathbb V}

% TO SHOW SOLUTIONS, include following (else comment out):
\newenvironment{soln}{
	\leavevmode\color{blue}\ignorespaces
}{}


\hypersetup{
	%    colorlinks,
	linkcolor={red!50!black},
	citecolor={blue!50!black},
	urlcolor={blue!80!black}
}

\geometry{
	top=1in,            % <-- you want to adjust this
	inner=1in,
	outer=1in,
	bottom=1in,
	headheight=3em,       % <-- and this
	headsep=2em,          % <-- and this
	footskip=3em,
}


\pagestyle{fancyplain}
\lhead{\fancyplain{}{Homework 1}}
\rhead{\fancyplain{}{CS 760 Machine Learning}}
\cfoot{\thepage}

\title{\textsc{Homework 1}} % Title

%%% NOTE:  Replace 'NAME HERE' etc., and delete any "\red{}" wrappers (so it won't show up as red)

\author{
	{Yichen Lin} \\
	{908 531 9599}\\
} 

\date{}

\begin{document}
	
	\maketitle 
	
	
	\textbf{Instructions:} 
	This is a background self-test on the type of math we will encounter in class. If you find many questions intimidating, we suggest you drop 760 and take it again in the future when you are more prepared.
	
	Use this latex file as a template to develop your homework.
	Submit your homework on time as a single pdf file to Canvas.
	There is no need to submit the latex source or any code.
	Please check Piazza for updates about the homework.
	
	
	\section{Vectors and Matrices [6 pts]}
	Consider the matrix $X$ and the vectors $\mathbf{y}$ and $\textbf{z}$ below:
	$$
	X = \begin{pmatrix}
		3 & 2 \\ -7 & -5 \\
	\end{pmatrix}
	\qquad \mathbf{y} = \begin{pmatrix}
		2 \\ 1
	\end{pmatrix} \qquad \mathbf{z} = \begin{pmatrix}
		1 \\ -1
	\end{pmatrix}
	$$
	\begin{enumerate}
		\item 	Computer $\mathbf{y}^{T} X \mathbf{z}$\\
			    \begin{soln}
					 $\mathbf{y}^{T} X \mathbf{z} \\
					 =\begin{pmatrix}
						 2 & 1 
					 \end{pmatrix} * 
					 \begin{pmatrix}
						 3 &2 \\ -7 &-5 \\
					 \end{pmatrix} * 
					 \begin{pmatrix}
						 1 \\ -1
					 \end{pmatrix} \\
					 = \begin{pmatrix}
						 1 & -1
					 \end{pmatrix} *
					 \begin{pmatrix}
						 1 \\ -1
					 \end{pmatrix} \\
					 = 0$   
				\end{soln}
		\item 	Is $X$ invertible? If so, give the inverse, and if no, explain why not.\\
		        \begin{soln}  The determinant of X is: \\
					$det(x) = 3*-5 - 2*-7 = -1 \neq 0$ \\
					So we can get the inverse matrix by: \\
					$\begin{pmatrix}
						3 &2 &1 &0 \\ -7 &-5 &0 &1 \\
					\end{pmatrix} \\
					\hookrightarrow \begin{pmatrix}
						1 &\frac{2}{3} &\frac{1}{3} &0 \\ 0 &-\frac{1}{3} &\frac{7}{3} &1 \\
					\end{pmatrix} \\
					\hookrightarrow \begin{pmatrix}
						1 &0 &5 &2 \\ 0 &1 &-7 &-3 \\
					\end{pmatrix}$ \\

					the inverse is $\begin{pmatrix}
						5 &2 \\ -7 &-3
					\end{pmatrix}$
				\end{soln}
	\end{enumerate}
	
	
	\section{Calculus [3 pts]}
	\begin{enumerate}
		\item If $y = e^{-x} + \arctan(z)x^{6/z} - \ln\cfrac{x}{x+1}$, what is the partial derivative of $y$ with respect to $x$?\\
		\begin{soln}
			$\frac{d y}{d x} = -e^{-x} + \arctan(z)*6/zx^{6/z-1} - \frac{1}{x}+ \frac{1}{x+1}$
		\end{soln}
	\end{enumerate}
	
	
	
	
	\section{Probability and Statistics [10 pts]}
	Consider a sequence of data $S = (1, 1, 1, 0, 1)$ created by flipping a coin $x$ five times, where 0 denotes that the coin turned up heads and 1 denotes that it turned up tails.
	\begin{enumerate}
		\item 	(2.5 pts) What is the probability of observing this data, assuming it was generated by flipping a biased coin with $p(x=1) = 0.6$?
		
		\begin{soln}
			$P(s) = 0.6 * 0.6 * 0.6 * 0.4 * 0.6 = 0.05184$
		\end{soln}
		
		\item 	(2.5 pts) Note that the probability of this data sample could be greater if the value of $p(x = 1)$ was not $0.6$, but instead some other value. What is the value that maximizes the probability of $S$? Please justify your answer.\\
		\begin{soln}
			The value that maximizes P(S) is $\frac{4}{5}$
			assume that $P(x=1) = p$ \\
			then $P(s) = p^{4}(1-p)$ \\
			$\frac{d P(s)}{d p} = 4p^{3}(1-p) - p^{4} = p^{3}(4-5p) = 0$\\
			$p = \frac{4}{5}$
		\end{soln}
		
		\item 	(5 pts) Consider the following joint probability table where both $A$ and $B$ are binary random variables: 
		\begin{table}[htb]
			\centering
			\begin{tabular}{ccc}\hline
				A & B & $P(A, B)$  \\\hline
				0 & 0 & 0.3 \\
				0 & 1 & 0.1 \\
				1 & 0 & 0.1 \\
				1 & 1 & 0.5 \\\hline
			\end{tabular}
		\end{table}
		\begin{enumerate}
			\item 	What is $P(A = 0 | B = 1)$?\\
			\begin{soln}  
				$P(A = 0 | B = 1) = \frac{P(A = 0, B = 1)}{P(B = 1)} = 01 / 0.5 = 0.2$ 
			\end{soln}
			 
			\item 	What is $P(A = 1 \vee B = 1 )$?\\
		     \begin{soln}  
				$P(A = 1 \vee B = 1 ) = 1 - P(A = 1\wedge B = 1) = 0.7$ 
			\end{soln}
		\end{enumerate}
	\end{enumerate}
	
	
	\section{Big-O Notation [6 pts]}
	For each pair $(f, g)$ of functions below, list which of the following
	are true: $f(n) = O(g(n))$, $g(n) = O(f(n))$, both, or
	neither. Briefly justify your answers.
	\begin{enumerate}
		\item 	$f(n) = \ln(n)$, $g(n) = \log_{2}(n)$.\\
		\begin{soln}
			$g(n) = O(f(n))$\\
			prove: \\
			assume :$h(n) = \log_{2}{n} - \ln(n)$, obviously $h(1) = 0$ \\
			$h(n)' = \frac{1}{n\ln(2)} - \frac{1}{n} = \frac{1}{n}(\frac{1}{\ln(2)}-1) > 0$ for all $n > 0$ \\
			for $n > 1$, $h(n) > 0$, $g(n) > f(n)$ 
		\end{soln}
		
		\item 	$f(n) =  \log_{2}\log_{2}(n)$, $g(n) = \log_{2}(n)$.\\
		\begin{soln}
			$g(n) = O(f(n))$\\
			prove: \\
			assume: $h(n) = \log_{2}{n} - \log_{2}{\log_{2}{n}}$, obviously $h(4) > 0$ \\
			$h(n)' = \frac{1}{n\ln_(2)} - \frac{1}{\ln_(2)\log_{2}{n}}\frac{1}{n\ln_(2)} = \frac{1}{n\ln_(2)}(1 - \frac{1}{\ln_(2)\log_{2}{n}})$ \\
			for all $n > 4$, $h(n)' > 0$, and$\frac{1}{\ln_(2)\log_{2}{n}}$ decreases with the increase of n\\
			so for all $n > 4$, $h(n)' > 0$, $h(n) > 0$, $g(n) > f(n)$

		\end{soln}
		
		\item 	$f(n) = n!$, $g(n) = 2^n$.\\
		\begin{soln}
			when $n = 4$, $f(n) = 24 > g(n) = 16$ \\
			when $n > 4$, $f(n) = n! \geqq 24 * 5^{n-4} > g(n) = 2^{n} = 16 * 2^{n-4}$ \\
			therefore, for all $n > 4$, $f(n) > g(n)$
			
		\end{soln}
	\end{enumerate}
	
	
	
	
	
	\section{Probability and Random Variables }
	\subsection{Probability [12.5 pts]}
	State true or false. Here $\Omega$ denotes the sample space and $A^c$ denotes the complement of the event $A$.
	\begin{enumerate}
		\item For any $A, B \subseteq \Omega$, $P(A|B)P(A) = P(B|A)P(B)$.\\
		\begin{soln}  True \end{soln}
		
		\item For any $A, B \subseteq \Omega$, $P(A \cup B) = P(A) + P(B) - P(B \cap A)$.\\         
		\begin{soln}  True \end{soln}
		
		\item For any $A, B, C \subseteq \Omega$ such that $P(B \cup C) > 0$,
		$\frac{P(A \cup B \cup C)}{P(B \cup C)} \geq P(A | B \cup C) P(B)$.\\ 
		\begin{soln}  True \end{soln}
		
		\item For any $A, B\subseteq\Omega$ such that $P(B) > 0, P(A^c) > 0$,
		$P(B|A^C) + P(B|A) = 1$.\\ 
		\begin{soln}  True \end{soln}
		
		\item If $A$ and $B$ are independent events, then $A^{c}$ and $B^{c}$ are independent.\\
		\begin{soln}  False \end{soln}
		
	\end{enumerate}
	
	\subsection{Discrete and Continuous Distributions [12.5 pts]}
	Match the distribution name to its probability density / mass
	function. Below, $|\xv| = k$.
	\begin{enumerate}[(a)]
		\begin{minipage}{0.3\linewidth}
			\item Gamma \begin{soln}  (j) \end{soln}
			\item Multinomial  \begin{soln}  (i)(g) \end{soln}
			\item Laplace \begin{soln}  (h) \end{soln}
			\item Poisson \begin{soln}  (I) \end{soln}
			\item Dirichlet  \begin{soln}  (k) \end{soln}
			
		\end{minipage}
		\begin{minipage}{0.5\linewidth}
			\item $f(\xv; \Sigmav, \muv) = \frac{1}{\sqrt{(2\pi)^k \mathrm{det}(\Sigmav) }} \exp\left( -\frac{1}{2}
			(\xv - \muv)^T \Sigmav^{-1} (\xv - \muv)  \right)$
			\item $f(x; n, \alpha) = \binom{n}{x} \alpha^x (1 - \alpha)^{n-x}$
			for $x \in \{0,\ldots, n\}$; $0$ otherwise
			\item $f(x; b, \mu) = \frac{1}{2b} \exp\left( - \frac{|x - \mu|}{b} \right)$
			\item $f(\xv; n, \alphav) = \frac{n!}{\Pi_{i=1}^k x_i!}
			\Pi_{i=1}^k \alpha_i^{x_i}$ for $x_i \in \{0,\ldots,n\}$ and
			$\sum_{i=1}^k x_i = n$; $0$ otherwise
			\item $f(x; \alpha, \beta) = \frac{\beta^{\alpha}}{\Gamma(\alpha)} x^{\alpha -
				1}e^{-\beta x}$ for $x \in (0,+\infty)$; $0$ otherwise
			\item $f(\xv; \alphav) = \frac{\Gamma(\sum_{i=1}^k
				\alpha_i)}{\prod_{i=1}^k \Gamma(\alpha_i)} \prod_{i=1}^{k}
			x_i^{\alpha_i - 1}$ for $x_i \in (0,1)$ and $\sum_{i=1}^k x_i =
			1$; 0 otherwise
			\item $f(x; \lambda) = \lambda^x \frac{e^{-\lambda}}{x!}$ for all
			$x \in Z^+$; $0$ otherwise
		\end{minipage}
	\end{enumerate}
	
	\subsection{Mean and Variance [10 pts]}
	\begin{enumerate}
		\item Consider a random variable which follows a Binomial
		distribution: $X \sim \text{Binomial}(n, p)$.
		\begin{enumerate}
			\item What is the mean of the random variable?\\
			\begin{soln}  $E(X) = np$ \end{soln}
			\item What is the variance of the random variable?\\
			\begin{soln}  $Var(X) = np(1-p)$ \end{soln}
		\end{enumerate}
		
		\item Let $X$ be a random variable and
		$\mathbb{E}[X] = 1, \Var(X) = 1$. Compute the following values:
		\begin{enumerate}
			\item $\mathbb{E}[5X]$\\
			\begin{soln}  $E(5X) = 5E(X) = 5$ \end{soln}
			\item $\Var(5X)$\\
			\begin{soln}  $Var(5X) = 25Var(X) = 25$ \end{soln}
			\item $\Var(X+5)$\\
			\begin{soln}  $Var(X+5) = Var(X) = 1$ \end{soln}
		\end{enumerate}
	\end{enumerate}
	
	%\clearpage
	
	\subsection{Mutual and Conditional Independence [12 pts]}
	\begin{enumerate}
		\item (3 pts) If $X$ and $Y$ are independent random variables, show that
		$\mathbb{E}[XY] = \mathbb{E}[X]\mathbb{E}[Y]$.
		
		\begin{soln}  
			Given that X and Y are independent \\
			$\E(XY) = \int_{+\infty}^{-\infty} xyf(x,y)dxdy$ \\
			$= \int_{+\infty}^{-\infty} xf_{X}(x)yf_{Y}(y)dxdy$ \\
			$= [\int_{+\infty}^{-\infty} xf_{X}(x)dx][\int_{+\infty}^{-\infty} yf_{Y}(y)dy]$ \\
			$= \E(x)\E(y)$ \\
		\end{soln}
		
		\item (3 pts) If $X$ and $Y$ are independent random variables, show that
		$\Var(X+Y) = \Var(X) + \Var(Y)$. \\
		Hint: $\Var(X+Y) = \Var(X) + 2\Cov(X, Y) + \Var(Y)$
		
		\begin{soln} 
			$\Var(X+Y) = \Var(X) + 2\Cov(X, Y) + \Var(Y)$ \\
			$\Cov(X,Y) = \E(X - \E(X))\E(Y - \E(Y))$ \\
			$= \E(XY) - \E(X)\E(Y) = 0$ \\
			Therefore, $|var(X+Y) = \Var(X) + \Var(Y)$ \\
		\end{soln}
		
		\item (6 pts) If we roll two dice that behave independently of each
		other, will the result of the first die tell us something about the
		result of the second die? 
		
		\begin{soln}  No. Because the result of the first die can't influence the second die \end{soln}
		
		If, however, the first die's result is a 1,
		and someone tells you about a third event --- that the sum of the two
		results is even --- then given this information is the result of the second die
		independent of the first die? 
		
		\begin{soln}
			No. Because the sum of the two result is even, which means the first result affects the second one. 
		\end{soln}
	\end{enumerate}
	
	\subsection{Central Limit Theorem [3 pts]}
	Prove the following result.
	\begin{enumerate}
		\item Let $X_i\sim\mathcal{N}(0, 1)$ and $\bar{X} = \frac{1}{n}\sum_{i=1}^n X_i$, then the distribution of $\bar{X}$ satisfies 
		$$\sqrt{n}\bar{X}\overset{n\rightarrow\infty}{\longrightarrow}\mathcal{N}(0, 1)$$
		
		\begin{soln}
			prove: \\  
			by central limit theorem, we have: $\frac{\sum_{i = 1}^{\infty} Xi }{\sqrt{n}} \sim\mathcal{N}(0, 1)$ \\
			therefore, $\lim_{n \to \infty}\frac{\sum_{i = 1}^{n}Xi}{\sqrt{n}} = \lim_{n \to \infty}\sqrt{n}\bar{X}\sim \mathcal{N}(0, 1)$
		\end{soln}
		
	\end{enumerate}
	
	
	
	\section{Linear algebra}
	
	
	\subsection{Norms [5 pts]}
	Draw the regions corresponding to vectors $\mathbf{x}\in\RR^2$ with the following norms:
	\begin{enumerate}
		\item 	$||\mathbf{x}||_1\leq 1$ (Recall that $||\mathbf{x}||_1 = \sum_i |x_i|$)

		\begin{soln}
	    % add figure filename, and remove % 
	    %    (this can be done by highlighting text and pressing "cmd + /" for sharelatex+mac)
	   	\begin{figure}[h!]
	       \centering
	       \includegraphics[width=0.4\textwidth]{Figure1.png}  
	               % reference folder/figure.pdf here and adjust width
	       \captionsetup{labelformat=empty}
	       \caption{}
	    \label{fig:my_label}
	   	\end{figure}
		\end{soln}
		
		\item 	$||\mathbf{x}||_2 \leq 1$ (Recall that $||\mathbf{x}||_2 =\sqrt{\sum_i x_i^2}$) 		
			\begin{soln}
	    	% add figure filename, and remove % 
	    	%    (this can be done by highlighting text and pressing "cmd + /" for sharelatex+mac)
	   		\begin{figure}[h!]
	        	\centering
	    	   	\includegraphics[width=0.4\textwidth]{Figure2.png}  
	               % reference folder/figure.pdf here and adjust width
	    	   	\captionsetup{labelformat=empty}
	    	   	\caption{}
	    	\label{fig:my_label}
	   		\end{figure}
		\end{soln}
		
		\item 	$||\mathbf{x}||_\infty \leq 1$ (Recall that $||\mathbf{x}||_\infty = \max_i |x_i|$) \\
			\begin{soln}
			% add figure filename, and remove % 
			%    (this can be done by highlighting text and pressing "cmd + /" for sharelatex+mac)
			\begin{figure}[h!]
			    \centering
			    \includegraphics[width=0.4\textwidth]{Figure3.png}  
			            % reference folder/figure.pdf here and adjust width
			    \captionsetup{labelformat=empty}
			    \caption{}
			    \label{fig:my_label}
			\end{figure}
		\end{soln}
	\end{enumerate}
	
	For $M = \begin{pmatrix}
		5 & 0 & 0 \\ 0 & 7 & 0 \\ 0 & 0 & 3
		
	\end{pmatrix}$, Calculate the following norms.
	\begin{enumerate}\addtocounter{enumi}{3}
		\item $||M||_{2}$ (L2 norm) \\
		\begin{soln}  $\mathbf{M}^\top\mathbf{M} = 
			\begin{pmatrix}
				25 & 0 & 0 \\ 0 & 49 & 0 \\ 0 & 0 & 9
			\end{pmatrix} 
			$ \\
			$\lambda = [9, 25, 49]$ \\
			$||M||_{2} = \sqrt{49} = 7$
		\end{soln}
		
		\item $||M||_{F}$ (Frobenius norm)\\
		\begin{soln}
			$||M||_{F} = \sqrt{tr(\mathbf{M}^\top\mathbf{M})} = 9.1104$
		\end{soln}
		
		
	\end{enumerate}
	
	
	
	\subsection{Geometry [10 pts]}
	Prove the following.  Provide all steps.
	\begin{enumerate}
		\item 	The smallest Euclidean distance from the origin to some point $\mathbf{x}$ in the hyperplane $\mathbf{w}^{T}\mathbf{x} + b = 0$ is $\frac{|b|}{||\mathbf{w}||_2}$.  You may assume $\mathbf{w} \neq 0$.\\
		\begin{soln} 
			assume that $\mathbf{x_{0}}$ is the projection of the origin to the hyperplane, which means $\mathbf{w}^{T}\mathbf{x_{0}} + b = 0$ \\
			then $\vec{x_{0}}$ is parallel with $\vec{w}$. \\
			so $||\vec{w}*\vec{x_{0}}|| = ||\mathbf{w}||_{2}*||\mathbf{x_{0}}||_{2} = ||\mathbf{w}||_{2}*d$ \\
			therefore, $||\mathbf{w}||_{2}*d = ||\mathbf{w}^{T}\mathbf{x_{0}}||_{2} = |b|$, and $d = \frac{|b|}{||\mathbf{w}||_2}$
		\end{soln}
		
		\item 	The Euclidean distance between two parallel hyperplane $\mathbf{w}^{T}\mathbf{x} + b_1 = 0$ and $\mathbf{w}^{T}\mathbf{x} + b_2 = 0$ is $\frac{|b_1 - b_2|}{||\mathbf{w}||_2}$ (Hint: you can use the result from the last question to help you prove this one).
		
		\begin{soln}
			assume that points $\mathbf{x_{1}}$ and $\mathbf{x_{2}}$ are on hyperplane $\mathbf{w}^{T}\mathbf{x} + b_1 = 0$ and $\mathbf{w}^{T}\mathbf{x} + b_2 = 0$ respectively. \\
			then those two points satisfy $\mathbf{w^{T}}\mathbf{x_{1}} + b_1 = 0$ and $\mathbf{w^{T}}\mathbf{x_{2}} + b_2 = 0$ \\
			and also, $||\mathbf{w}||_2d = ||\vec{w}*\vec{x_2x_1}||_2 = ||\mathbf{w^{T}}(\mathbf{x_1}-\mathbf{x_2})||_2 = |b_1 - b_2|$ \\
			so $d = \frac{|b_1 - b_2|}{||\mathbf{w}||_2}$ 
		\end{soln}
		
	\end{enumerate}
	
	
	
	\section{Programming Skills [10 pts]}
	Sampling from a distribution.  For each question, submit a scatter plot (you will have 2 plots in total).  Make sure the axes for all plots have the same ranges.
	\begin{enumerate}
		\item Make a scatter plot by drawing 100 items from a two dimensional Gaussian $N((1, -1)^{T}, 2I)$, where I is an identity matrix in $\mathbb{R}^{2 \times 2}$.
		
			\begin{soln}
			% add figure filename, and remove % 
			%    (this can be done by highlighting text and pressing "cmd + /" for sharelatex+mac)
			\begin{figure}[h!]
			    \centering
			    \includegraphics[width=0.4\textwidth]{Figure4.png}  
			            % reference folder/figure.pdf here and adjust width
			    \captionsetup{labelformat=empty}
			    \caption{}
			    \label{fig:my_label}
			\end{figure}
		\end{soln}
	
		\item Make a scatter plot by drawing 100 items from a mixture distribution 
		$0.3 N\left((5, 0)^{T}, \begin{pmatrix} 1 & 0.25 \\ 0.25 & 1\\ \end{pmatrix}\right)
		+0.7 N\left((-5, 0)^{T}, \begin{pmatrix} 1 & -0.25 \\ -0.25 & 1\\ \end{pmatrix}\right)
		$.
		
		\begin{soln}
		% add figure filename, and remove % 
		%    (this can be done by highlighting text and pressing "cmd + /" for sharelatex+mac)
		\begin{figure}[h!]
		    \centering
		    \includegraphics[width=0.4\textwidth]{Figure5.png}  
		            % reference folder/figure.pdf here and adjust width
		    \captionsetup{labelformat=empty}
		    \caption{}
		    \label{fig:my_label}
		\end{figure}
	\end{soln}
	\end{enumerate}
	
	
	\bibliographystyle{apalike}
	
	
	%----------------------------------------------------------------------------------------
	
	
\end{document}
